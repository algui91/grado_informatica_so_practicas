% This is a sample file for the Undergraduate Faculty Program at
% PCMI, containing one lecture
%
%   What is a Partial Differential Equation?
%
%  from the Park City Lectures of Andrew J. Bernoff.
%
% The sample file illustrates the use of epsf.tex to include postscript
% graphics, as well as various "AmS-LaTeX" constructions from the amsmath
% package (which is automatically loaded by the pcms-l document class).
% To run this file you need the files
%
%        pcms-l.cls and pcmslmod.tex
%


\documentclass[lecture,10pt,]{pcms-l}
% Filename: pcmslmod.tex v.1.4

% Contains modifications to pcms-l.cls (version 1.2d, 1997/01/02)

% Written by David R. Morrison, 1997/04/20; revised 1997/11/23; 1998/2/9;
% 1998/3/13; 1998/3/31; 1998/5/29
\typeout{pcmslmod.tex v.1.4, 1998/5/29}


% Summary of changes
%
% 1. A new command \lectureseries, which specifies the title of the lecture
% series and does the page setup for the first page of the series
% (eliminating the need for a \chapter*{} command).  An optional argument
% allows the specification of a shortened title for running heads.
%
% 2. A modification of the \auth command, allowing the specification of a
% shortened author name for running heads.
%
% 3. A modification of the \lecture command, giving it an argument for the
% specification of the lecture name (rather than a separate \lecturename
% command), and incorporating the \chapter*{} command so that the latter
% does not need to be invoked by the user.  From the user's perspective,
% \lecture should function in a very similar fashion to \chapter.
%
% 3A. In v.1.1 of this file, an alternate "starred" form \lecture* is
% introduced, which allows for the inclusion of unnumbered lectures (such
% as a preface, or a list of problems).  The title of this unnumbered
% lecture is flush left if the lecture is the first one (i.e., the
% preface or introduction), otherwise it is set flush right.
%
% 4. Two style corrections in running heads: the body of the running heads
% should be set in medium weight rather than bold, and the author and title
% should be separated by a comma rather than a period.
%
% 5. Modifications of bibliography and index commands, so that their
% headings and running heads have the same style as lectures.  A new
% user-specifiable option \ifBibliographyIsASection (with default value
% \BibliographyIsASectionfalse) is introduced: it should be set to true if
% a user wants separate bibliography sections at the end of each lecture
% rather than a single bibliography at the end.
%
% 6. A minor style change: since the footnote giving the author address is
% typically several lines long, with subsequent lines giving information
% such as e-mail or current addresses which has equal logical weight to the
% address itself, having the first line of the footnote indented presents a
% strange appearance.  As an easy fix, indentation from footnotes was
% removed. 
%
% 6A. In v.1.1 of this file, the commands \thanks, \subjclass, \keywords,
% and \date were all implemented; as in most AMS styles, they produce 
% additional footnotes on the first page.
%
% 7. The sample files have also been rewritten in a way compatible with these
% changes. 
%
% 8. Changes in v 1.4: fixed font size of author, and fixed spacing after
% lecture 

\makeatletter

% First, we define a new command \lectureseries, replacing the 
% \chapter*{title} command at the head of the file.  An optional argument
% allows a shortened form to be specified for use in running heads.  (The
% thing used in running heads is \thelectureseries -- this is unchanged.)
%
% NB: \part* and \pauth commands could still be used to generate a separate
% title page for an individual lecture series, if that is desired.
%
% We also introduce \iffirstlecture, which is set to true by the
% \lectureseries command so that the top-of-page formatting is not repeated
% by the \lecture command in this case.

\newif\iffirstlecture\firstlecturefalse

\newcommand{\lectureseries}{\firstlecturetrue
              \secdef\@lectureseries\@slectureseries} 

\newcommand{\@lectureseries}[2][default]{\chapter*{#2}%
              \gdef\thelectureseries{#1}} 

\newcommand{\@slectureseries}[1]{\chapter*{#1}}

% Next, we redefine \auth to allow for the specification of a shortened
% author name in running heads, as an optional argument.  (TeX-nical note:
% it might have been better to write this command and the previous one
% using \@dblarg rather than \secdef.)

\renewcommand{\auth}{\secdef\@auth\@sauth}

\newcommand{\@auth}[2][default]{\vspace{-1pc}{\raggedleft
        \Large\bf\noindent
        #2\endgraf
        \vspace*{2pc}
        }
        \def\@author{#1}%
}

\newcommand{\@sauth}[1]{\vspace{-1pc}{\raggedleft
        \Large\bf\noindent
        #1\endgraf
        \vspace*{2pc}
        }
        \def\@author{#1}%
}

% Next, we redefine \lecture so that \chapter*{} is not needed.  To make
% this work, we only want to insert \chapter*{} starting with the second
% \lecture command, which was why we introduced \iffirstlecture.
%
% Also, \lecture now takes an argument specifying the title, in place of
% the old \lecturename command.

\def\lecture#1{\global\Lecturetrue\global\Monographfalse
\iffirstlecture\else\chapter*{}\fi\firstlecturefalse
  \global\let\sectionmark\@gobble % \lecturemark will be used instead
  \addtocounter{lecture}1\relax
  \refstepcounter{chapter}%
%  \addtocounter{chapter}1\relax % this is done for section numbering
\gdef\thelecturename{#1\unskip}
  {\Large\bfseries
   \raggedleft
   \@xp\uppercase\@xp{\thelecturelabel} {\LARGE\thelecturenum}\\
   \vspace*{3pt}%
   \thelecturename
   \endgraf}%
  \let\@secnumber=\thelecturenum
  \@xp\lecturemark\@xp{\thelecturename}%
  \addcontentsline{toc}{chapter}{%
    \thelecturelabel\ \thelecturenum.\ \thelecturename}%
  \vspace{34\p@}\noindent}
  
% In v.1.1, \lecture is redefined again, to implement the inclusion of a
% "starred" form \lecture*.

\def\lecture{\global\Lecturetrue\global\Monographfalse
\iffirstlecture\else\chapter*{}\fi%
  \global\let\sectionmark\@gobble % \lecturemark will be used instead
\secdef\@lecture\@slecture}

\def\@lecture[#1]#2{%
  \addtocounter{lecture}1\relax
  \refstepcounter{chapter}%
%  \addtocounter{chapter}1\relax % this is done for section numbering
\gdef\thelecturename{#1\unskip}\firstlecturefalse
  {\Large\bfseries
   \raggedleft
   \@xp\uppercase\@xp{\thelecturelabel} {\LARGE\thelecturenum}\\
   \vspace*{3pt}%
%   \thelecturename
    #2\unskip
   \endgraf}%
  \let\@secnumber=\thelecturenum
  \@xp\lecturemark\@xp{\thelecturename}%
  \addcontentsline{toc}{chapter}{%
%    \thelecturelabel\ \thelecturenum.\ \thelecturename}%
    \thelecturelabel\ \thelecturenum.\ #2}%
  \vspace{34\p@}\noindent}
  
\def\slecturerunhead#1#2#3{%
    \let\@tempa\chaptername
    \uppercasenonmath{\@tempa}%
    \def\@tempb{#3\unskip}%
    \uppercasenonmath{\@tempb}%
    {\normalfont\@tempb}
    }
\def\slecturemark{%\let\@secnumber\@empty
%    \@secmark\markright\sectionrunhead\sectionname}%
    \@secmark\markright\slecturerunhead\chaptername}%


\def\@slecture#1{%
\iffirstlecture
%  \addtocounter{lecture}1\relax
%  \refstepcounter{chapter}%
%%  \addtocounter{chapter}1\relax % this is done for section numbering
\gdef\thelecturename{#1\unskip}\firstlecturefalse
  {\Large\bfseries
%   \raggedleft
%   \@xp\uppercase\@xp{\thelecturelabel} {\LARGE\thelecturenum}\\
%   \vspace*{3pt}%
%\noindent\@xp\uppercase\@xp{\thelecturename} 
\noindent\thelecturename
   \endgraf}%
  \let\@secnumber=\thelecturenum
  \@xp\slecturemark\@xp{\thelecturename}%
%\markright\thelecturename
  \addcontentsline{toc}{chapter}{%
    \thelecturename}%
 \vspace{-6\p@}\noindent
%\noindent
\else
%  \addtocounter{lecture}1\relax
%  \refstepcounter{chapter}%
%%  \addtocounter{chapter}1\relax % this is done for section numbering
\gdef\thelecturename{#1\unskip}\firstlecturefalse
  {\Large\bfseries
   \raggedleft
%   \@xp\uppercase\@xp{\thelecturelabel} {\LARGE\thelecturenum}\\
%   \vspace*{3pt}%
   \@xp\uppercase\@xp{\thelecturename}
   \endgraf}%
  \let\@secnumber=\thelecturenum
  \@xp\slecturemark\@xp{\thelecturename}%
%\markright\thelecturename
  \addcontentsline{toc}{chapter}{%
    \thelecturename}%
  \vspace{34\p@}\noindent
\fi}

% We make the following changes to definitions of running heads: 
% (1) add \textmd so that the header is not boldface
% (2) use a comma, not a period, to separate author and lectureseries

\ifLecture
  \def\chapterrunhead#1#2#3{%
    \let\@tempa\@author
    \uppercasenonmath{\@tempa}%
    \uppercasenonmath{\thelectureseries}%
    \textmd{\@tempa, \thelectureseries}
    }
  \def\lecturerunhead#1#2#3{%
    \let\@tempa\chaptername
    \uppercasenonmath{\@tempa}%
    \def\@tempb{#3\unskip}%
    \uppercasenonmath{\@tempb}%
    \textmd{\@tempa\ #2. \@tempb}
    }
\else
  \let\chapterrunhead\partrunhead
\fi

% For the bibliography, we do two things
% (1) we introduce \ifBibliographyIsASection (default is false) to decide
% if a section or a chapter.  When its a chapter, but we are in lecture
% mode, we use the lecture style of headings.  If its a section, it should
% NOT be in backmatter.
% (2) we fix the running heads to be consistent with everything else.

\newif\ifBibliographyIsASection\BibliographyIsASectionfalse

  \def\bibliomark{%\let\@secnumber\@empty
%    \@secmark\markright\sectionrunhead\sectionname}%
    \@secmark\markright\bibliorunhead\chaptername}%

  \def\bibliorunhead#1#2#3{%
    \let\@tempa\chaptername
    \uppercasenonmath{\@tempa}%
    \def\@tempb{#3\unskip}%
    \uppercasenonmath{\@tempb}%
    \textmd{\@tempb}
    }

\def\thebibliography#1{%
  \ifBibliographyIsASection
    \section*\refname
    \if@backmatter
      \markboth{\refname}{\refname}%
    \fi
  \else
\chapter*{}
  {\Large\bfseries
   \raggedleft
   \@xp\uppercase\@xp{\bibname} \\
   \endgraf}%
  \let\@secnumber=\thelecturenum
  \@xp\bibliomark\@xp{\bibname}%
  \addcontentsline{toc}{chapter}{%
    \bibname}%
  \vspace{34\p@}\noindent
  \fi
  \normalsize\labelsep .5em\relax
  \list{\@arabic\c@enumi.}{\settowidth\labelwidth{\@biblabel{#1}}%
  \leftmargin\labelwidth
  \advance\leftmargin\labelsep
%	\bibsetup\relax
	\usecounter{enumi}}\sloppy
  \clubpenalty9999 \widowpenalty\clubpenalty  \sfcode`\.\@m}

% We also want to change the headings and running heads for the index.  We
% only do this in the case of a lecture (so the previous definition will still
% be invoked in the case of a monograph.)

  \def\indexmark{%\let\@secnumber\@empty
%    \@secmark\markright\sectionrunhead\sectionname}%
    \@secmark\markright\indexrunhead\chaptername}%

  \def\indexrunhead#1#2#3{%
    \let\@tempa\chaptername
    \uppercasenonmath{\@tempa}%
    \def\@tempb{#3\unskip}%
    \uppercasenonmath{\@tempb}%
    \textmd{\@tempb}
    }

\ifLecture
\def\theindex{\cleardoublepage
\@restonecoltrue\if@twocolumn\@restonecolfalse\fi
\columnseprule \z@ \columnsep 35pt
\def\indexchap{\@startsection
		{chapter}{1}{\z@}{8pc}{34pt}%
		{\raggedleft
		\Large\bfseries}}%
 \twocolumn[\indexchap[{\indexname}]{\@xp\uppercase\@xp{\indexname}}]
%		\Large\bfseries}}%
% \twocolumn[\indexchap*{\@xp\uppercase\@xp{\indexname}}]
% \@mkboth{{\indexname}}{{\indexname}}%
  \@xp\indexmark\@xp{\indexname}%
	\thispagestyle{plain}\let\item\@idxitem\parindent\z@
	 \footnotesize\parskip\z@ plus .3pt\relax\let\item\@idxitem}
\fi

% Finally, a small stylistic change: for the footnote giving the author
% address, indenting the footnote doesn't look good (in my opinion) due to
% the email line NOT being indented. So we change:
%
% \def\@makefntext{\indent\@makefnmark}
%
% to

\def\@makefntext{\noindent\@makefnmark}

% In v.1.1, we also implement \thanks and other commands which make
% first-page footnotes:

\def\setaddress{%
  {\let\@makefnmark\relax  \let\@thefnmark\relax
        \nobreak
        \addressnum@=\z@
        \loop\ifnum\addressnum@<\addresscount@\advance\addressnum@\@ne
           \footnote{$^{\hbox{\tiny\number\addressnum@}}$%
           \csname @address\number\addressnum@\endcsname
           \csname @curraddr\number\addressnum@\endcsname
           \csname @email\number\addressnum@\endcsname}\repeat
  \ifx\@empty\@date\else \@footnotetext{\@setdate}\fi
  \ifx\@empty\@subjclass\else \@footnotetext{\@setsubjclass}\fi
  \ifx\@empty\@keywords\else \@footnotetext{\@setkeywords}\fi
  \ifx\@empty\thankses\else \@footnotetext{%
    \def\par{\let\par\@par}\@setthanks}\fi
    }%
  \@setcopyright
}

% fix blank pages (Dan Freed -- November 25, 1997)

\def\@tmpevenhead{\relax}

\def\cleardoublepage{\clearpage\if@twoside \ifodd\c@page\else
    \let\@tmpevenhead\@evenhead \let\@evenhead\relax\hbox{}\eject 
    \let\@evenhead\@tmpevenhead\if@twocolumn\hbox{}\newpage\fi\fi\fi}

% define \copyrightyear to be \currentyear (Dan Freed -- March 13, 1998)

\def\@setcopyright{%
  \let\copyrightyear\currentyear             % DF
  \insert\copyins{\hsize\textwidth
    \parfillskip\z@ \leftskip\z@\@plus.9\textwidth
    \fontsize{6}{7\p@}\normalfont\upshape
    \everypar{}%
    \vskip-\skip\copyins \nointerlineskip
    \noindent\vrule\@width\z@\@height\skip\copyins
    \copyright\copyrightyear\ elbauldelprogramador.com\par
    \kern\z@}%
}

% macro to put in blank page at beginning for 2-up viewing

\def\BlankPage{\pagestyle{empty}\thispagestyle{empty}\null\vfil\eject} 

%fix font size of author:

\renewcommand{\@auth}[2][default]{{\raggedleft
        \begingroup
  \fontsize{\@xivpt}{18}\bfseries%\centering
  #2\par \endgroup
        \vspace*{2pc}
        }
        \def\@author{#1}%
}

\renewcommand{\@sauth}[1]{{\raggedleft
        \begingroup
  \fontsize{\@xivpt}{18}\bfseries%\centering
  #1\par \endgroup
        \vspace*{2pc}
        }
        \def\@author{#1}%
}

%fix spacing after lecture

\def\@lecture[#1]#2{%
  \addtocounter{lecture}1\relax
  \refstepcounter{chapter}%
\gdef\thelecturename{#1\unskip}\firstlecturefalse
  {\Large\bfseries
   \raggedleft
   \@xp\uppercase\@xp{\thelecturelabel} {\LARGE\thelecturenum}\\
   \vspace*{3pt}%
    #2\unskip
   \endgraf}%
  \let\@secnumber=\thelecturenum
  \@xp\lecturemark\@xp{\thelecturename}%
  \addcontentsline{toc}{chapter}{%
    \thelecturelabel\ \thelecturenum.\ #2}%
  \vspace{10\p@}\noindent}
  
\def\@slecture#1{%
\iffirstlecture
\gdef\thelecturename{#1\unskip}\firstlecturefalse
  {\Large\bfseries
\noindent\thelecturename
   \endgraf}%
  \let\@secnumber=\thelecturenum
  \@xp\slecturemark\@xp{\thelecturename}%
  \addcontentsline{toc}{chapter}{%
    \thelecturename}%
 \vspace{-6\p@}\noindent
\else
\gdef\thelecturename{#1\unskip}\firstlecturefalse
  {\Large\bfseries
   \raggedleft
   \@xp\uppercase\@xp{\thelecturename}
   \endgraf}%
  \let\@secnumber=\thelecturenum
  \@xp\slecturemark\@xp{\thelecturename}%
  \addcontentsline{toc}{chapter}{%
    \thelecturename}%
  \vspace{10\p@}\noindent
\fi}



% End of modifications to pcms-l.cls.

\makeatother
  % v.1.2
%\input epsf.tex
\usepackage[margin=3cm]{geometry}
%\usepackage{amssymb} % this command would have loaded all the extra symbols,



% authors should not define these, they will be defined by the volume editors
%\def\currentvolume{3}
%\def\currentyear{1993}

\usepackage{amsmath}
\usepackage{breqn}  %Para separar las ecuaciones largas
\usepackage{enumerate} %Enumerar con numeros romanos
% EQUATION NUMBERING AND THEOREM SETUP

\numberwithin{section}{chapter}
\numberwithin{equation}{chapter}

\theoremstyle{plain}
\newtheorem{theorem}[equation]{Theorem}
\newtheorem{lemma}[equation]{Lemma}

\theoremstyle{definition}
\newtheorem{definition}[equation]{Definition}

\newtheorem{exercise}{Exercise}
\newtheorem{example}{Ejemplo}
\newtheorem{problem}{Problem}
\newtheorem*{remark}{Remark}

% Set enumerate to use letters, not numbers for problem parts.

\renewcommand{\theenumi}{\alph{enumi}}
\renewcommand{\labelenumi}{(\theenumi)}

% AUTHOR-DEFINED MACROS:

\newif\ifEdicionPC % Por defecto vale false

% AUTHOR-DEFINED MACROS:
\newif\ifEdicionPC % Por defecto vale false

%Descomentar para hacer valor veradero
\EdicionPCtrue

\ifEdicionPC
    %Fuente Sans-serif para leer en pc
    \usepackage{fontspec} % Allows font customization
    \defaultfontfeatures{Mapping=tex-text,Scale=MatchLowercase}
    \setmainfont{Ubuntu Light} % Main document font
    \setmonofont{Ubuntu Mono}

\else
    %Fuente para Imprimir en papel, serif
    %\usepackage{txfonts}
    \usepackage{fontspec} % Allows font customization
    \defaultfontfeatures{Mapping=tex-text,Scale=MatchLowercase}
    \setmainfont{Lucida} % Main document font
    \setmonofont{Ubuntu Mono}
\fi
%%%%%%%%%%%%%%%%%%%%%%%%%%%%%%%%%%%%%%%%%
% Stylish Colored Title Page 
% LaTeX Template
% Version 1.0 (27/12/12)
%
% This template has been downloaded from:
% http://www.LaTeXTemplates.com
%
% Original author:
% Peter Wilson (herries.press@earthlink.net)
%
% License:
% CC BY-NC-SA 3.0 (http://creativecommons.org/licenses/by-nc-sa/3.0/)
% 
% Instructions for using this template:
% This title page compiles as is. If you wish to include this title page in 
% another document, you will need to copy everything before 
% \begin{document} into the preamble of your document. The title page is
% then included using \titleBC within your document.
%
%%%%%%%%%%%%%%%%%%%%%%%%%%%%%%%%%%%%%%%%%

%----------------------------------------------------------------------------------------
%	PACKAGES AND OTHER DOCUMENT CONFIGURATIONS
%----------------------------------------------------------------------------------------

%\documentclass{book}

\usepackage[svgnames]{xcolor} % Required to specify font color

\newcommand*{\plogo}{\fbox{$\mathcal{ELBAULDELPROGRAMADOR}$}} % Generic publisher logo

\usepackage{graphicx} % Required for box manipulation
\usepackage{anyfontsize}
\usepackage{hyperref}
%----------------------------------------------------------------------------------------
%	TITLE PAGE
%----------------------------------------------------------------------------------------

\newcommand*{\rotrt}[1]{\rotatebox{90}{#1}} % Command to rotate right 90 degrees
\newcommand*{\rotlft}[1]{\rotatebox{-90}{#1}} % Command to rotate left 90 degrees

\newcommand*{\titleBC}{\begingroup % Create the command for including the title page in the document
\centering % Center all text

\def\titleSize{41}
\def\carreraSize{20}
\def\escuelaSize{13}
\def\ciudadSize{10}

%\def\CP{\textit{\Huge Apuntes de Algorítmica}} % Title
\def\titulo{\textit{\fontsize{\titleSize}{50}\selectfont Ejercicios Prácticas SO}}
\def\carrera{\fontsize{\carreraSize}{50}\selectfont Grado Ingeniería Informática}
\def\escuela{\fontsize{\escuelaSize}{50}\selectfont E.T.S. Ing. Informática y de Telecomunicación (ETSIIT)}
\def\ciudad{\fontsize{\ciudadSize}{50}\selectfont Granada}

\settowidth{\unitlength}{\titulo} % Set the width of the curly brackets to the width of the title
{\color{LightGoldenrod}\resizebox*{\unitlength}{\baselineskip}{\rotrt{$|$}}} \\[\baselineskip] % Print top curly bracket
\textcolor{Sienna}{\titulo} \\[\baselineskip] % Print title
{\color{RosyBrown}\carrera} \\[\baselineskip] % Tagline or further description
{\color{RosyBrown}\escuela} \\[\baselineskip] % Tagline or further description
{\color{RosyBrown}\Small Granada} \\ % Tagline or further description
{\color{LightGoldenrod}\resizebox*{\unitlength}{\baselineskip}{\rotlft{$|$}}} % Print bottom curly bracket

\vfill % Whitespace between the title and the author name

{\Large\textbf{Alejandro Alcalde}}\\ % Author name
{\Large\href{http://elbauldelprogramador.com}{elbauldelprogramador.com}}

\vfill % Whitespace between the author name and the publisher logo

%\plogo\\[0.5\baselineskip] % Publisher logo
\date{\today} % Year published

\endgroup}

%----------------------------------------------------------------------------------------
%	BLANK DOCUMENT
%----------------------------------------------------------------------------------------

%\begin{document} 

%\pagestyle{empty} % Removes page numbers

%\titleBC % This command includes the title page

%\end{document}
\usepackage{minted}

\newmintedfile[myBash]{bash}{
    linenos,
    numbersep=5pt,
    gobble=0,
    frame=lines,
    framesep=2mm,
}

\newcommand{\bashscript}[2]{
    \myBash[label=#2.sh]{#1.sh}
}

\newmintedfile[myC]{cpp}{
    linenos,
    numbersep=5pt,
    gobble=0,
    frame=lines,
    framesep=2mm,
}

\newcommand{\cppscript}[2]{
    \myC[label=#2.c]{#1.c}
}

\newminted{bash}{
   gobble=0,
   frame=lines,
   framesep=2mm,
   mathescape=true,
}

\newminted{cpp}{
   gobble=0,
   frame=lines,
   framesep=2mm,
   mathescape=true,
}

\begin{document}

\mainmatter
\setcounter{page}{1}

%\LogoOn
\thispagestyle{empty}
\titleBC

\lectureseries[Ejercicios sesiones Prácticas]{Ejercicios sesiones Prácticas}
\auth{\href{http://elbauldelprogramador.com}{Alejandro Alcalde}}
\address{ETSIIT, Granada}
\email{algui91@gmail.com}

%The following items will become first page footnotes; they are optional.
\date{\today}
\setaddress

% the following hack starts the lecture numbering at 1
\setcounter{lecture}{1}
\setcounter{chapter}{1}

%%%%%%%%%%%%%%%%%%%%%%%%%%%%%%%%%%%%%%%%%%%%%%%%
%%%%%%%%%%%%%%%%% REpaso T1 T2 FS %%%%%%%%%%%%%%
%%%%%%%%%%%%%%%%%%%%%%%%%%%%%%%%%%%%%%%%%%%%%%%%

%\chapter{Sesión 1}

\begin{exercise}
?`Qué hace el siguiente programa? Probad tras la ejecución del programa las
siguientes órdenes del shell: \verb!$>cat archivo y $> od -c archivo!
\begin{cppcode}
/*
tarea1.c
Trabajo con llamadas al sistema del Sistema de Archivos ''POSIX 2.10 compliant''
Probar tras la ejecución del programa: >cat archivo y > od -c archivo
*/

#include<sys/types.h>   //Primitive system data types for abstraction of implementation-dependent data types.
                        //POSIX Standard: 2.6 Primitive System Data Types <sys/types.h>
#include<sys/stat.h>
#include<fcntl.h>
#include<stdlib.h>
#include<stdio.h>
#include<errno.h>

char buf1[]="abcdefghij";
char buf2[]="ABCDEFGHIJ";

int main(int argc, char *argv[])
{
int fd;

if( (fd=open("archivo",O_CREAT|O_TRUNC|O_WRONLY,S_IRUSR|S_IWUSR))<0) {
    printf("\nError %d en open",errno);
    perror("\nError en open");
    exit(-1);
}
if(write(fd,buf1,10) != 10) {
    perror("\nError en primer write");
    exit(-1);
}

if(lseek(fd,40,SEEK_SET) < 0) {
    perror("\nError en lseek");
    exit(-1);
}

if(write(fd,buf2,10) != 10) {
    perror("\nError en segundo write");
    exit(-1);
}

return 0;
}
\end{cppcode}

{\color{blue} El programa crea un archivo llamado \verb!archivo!, escribe el primer buffer, que contiene \verb!abcdefghij!, éstos caracteres se escriben desde la primera posición del archivo, luego, mueve el puntero a la posición 40 y escribe el segundo buffer, que contiene \verb!ABCDEFGHIJ!, por tanto, el fichero va a tener contenido nulo en las posiciones 11-39. Con el comando \verb!cat! no se aprecia el contenido nulo porque es ignodaro, sin embargo con od sí se muestran:}

\begin{bashcode}
 od -c archivo 
0000000   a   b   c   d   e   f   g   h   i   j  \0  \0  \0  \0  \0  \0
0000020  \0  \0  \0  \0  \0  \0  \0  \0  \0  \0  \0  \0  \0  \0  \0  \0
0000040  \0  \0  \0  \0  \0  \0  \0  \0   A   B   C   D   E   F   G   H
0000060   I   J
0000062
\end{bashcode}

\end{exercise}

\begin{exercise}
Implementa un programa que acepte como argumento un ''pathname'', abra el
archivo correspondiente y utilizando un tamaño de lectura en bloques de 80 Bytes cree un
archivo de salida en el que debe aparecer lo siguiente:
\begin{bashcode}
Bloque 1
//los primeros 80 Bytes
Bloque 2
//los siguientes 80 Bytes
...
Bloque n
//los siguientes 80 Bytes
\end{bashcode}
\cppscript{../Sesion1/src/ej2}{Ejercicio 2}

?`Cómo tendrías que modificar el programa para que una vez
finalizada la escritura en el archivo de salida y antes de cerrarlo, pudiésemos indicar en su
primera línea el número de etiquetas ''Bloque i'' escritas de forma que tuviese la siguiente
apariencia?:

{\color{blue} Implementado en la versión de arriba}
\end{exercise}

\begin{exercise}
?`Qué hace el siguiente programa?

\begin{cppcode}
/*
tarea2.c
Trabajo con llamadas al sistema del Sistema de Archivos ''POSIX 2.10 compliant''
*/

#include<sys/types.h>    //Primitive system data types for abstraction of implementation-dependent data types.
                        //POSIX Standard: 2.6 Primitive System Data Types <sys/types.h>
#include<unistd.h>        //POSIX Standard: 2.10 Symbolic Constants         <unistd.h>
#include<sys/stat.h>
#include<stdio.h>
#include<errno.h>
#include<string.h>

int main(int argc, char *argv[])
{
int i;
struct stat atributos;
char tipoArchivo[30];

if(argc<2) {
    printf("\nSintaxis de ejecucion: tarea2 [<nombre_archivo>]+\n\n");
    exit(-1);
}
for(i=1;i<argc;i++) {
    printf("%s: ", argv[i]);
    if(lstat(argv[i],&atributos) < 0) {
        printf("\nError al intentar acceder a los atributos de %s",argv[i]);
        perror("\nError en lstat");
    }
    else {
        if(S_ISREG(atributos.st_mode)) strcpy(tipoArchivo,"Regular");
        else if(S_ISDIR(atributos.st_mode)) strcpy(tipoArchivo,"Directorio");
        else if(S_ISCHR(atributos.st_mode)) strcpy(tipoArchivo,"Especial de caracteres");
        else if(S_ISBLK(atributos.st_mode)) strcpy(tipoArchivo,"Especial de bloques");
        else if(S_ISFIFO(atributos.st_mode)) strcpy(tipoArchivo,"Tuber�a con nombre (FIFO)");
        else if(S_ISLNK(atributos.st_mode)) strcpy(tipoArchivo,"Enlace relativo (soft)");
        else if(S_ISSOCK(atributos.st_mode)) strcpy(tipoArchivo,"Socket");
        else strcpy(tipoArchivo,"Tipo de archivo desconocido");
        printf("%s\n",tipoArchivo);
    }
}

return 0;
}
\end{cppcode}

{\color{blue} 
Para cada archivo que se le pase como parámetro, muestra qué tipo de ficheros son, en este caso usando la llamada \verb/lstat/, que se diferencia de \verb/stat/ en que si el archivo es un enlace simbólico, \verb/lstat/ proporciona los atributos del enlace en sí, no del archivo al que referencia, ejemplo de ejecución:

\begin{bashcode}
./tarea2 *
dump: Regular
ej2: Regular
ej2.c: Regular
ej2.c~: Regular
fichero: Regular
fichero~: Regular
tarea1.c: Regular
tarea2: Regular
tarea2.c: Regular
tarea3.c: Regular
\end{bashcode}
}
\end{exercise}

\begin{exercise}
Define una macro en lenguaje C que tenga la misma funcionalidad que la macro
S_ISREG(mode) usando para ello los flags definidos en <sys/stat.h> para el campo st_mode
de la struct stat, y comprueba que funciona en un programa simple. Consulta en un libro
de C o en internet cómo se especifica una macro con argumento en C.

\cppscript{../Sesion1/src/ej4}{Ejercicio 4}
\end{exercise}/
%\chapter{Sesión 2}

\begin{exercise}
?`Qué hace el siguiente programa?

\end{exercise}
\begin{cppcode}
/*
tarea3.c
Trabajo con llamadas al sistema del Sistema de Archivos ''POSIX 2.10 compliant''
Este programa fuente está pensado para que se cree primero un programa con la parte
 de CREACION DE ARCHIVOS y se haga un ls -l para fijarnos en los permisos y entender
 la llamada umask.
En segundo lugar (una vez creados los archivos) hay que crear un segundo programa
 con la parte de CAMBIO DE PERMISOS para comprender el cambio de permisos relativo
 a los permisos que actualmente tiene un archivo frente a un establecimiento de permisos
 absoluto.
*/

#include<sys/types.h>   //Primitive system data types for abstraction of implementation-dependent data types.
                        //POSIX Standard: 2.6 Primitive System Data Types <sys/types.h>
#include<unistd.h>      //POSIX Standard: 2.10 Symbolic Constants         <unistd.h>
#include<sys/stat.h>
#include<fcntl.h>       //Needed for open
#include<stdio.h>
#include<errno.h>


int main(int argc, char *argv[])
{
int fd1,fd2;
struct stat atributos;

//CREACION DE ARCHIVOS
if( (fd1=open("archivo1",O_CREAT|O_TRUNC|O_WRONLY,S_IRGRP|S_IWGRP|S_IXGRP))<0) {
    printf("\nError %d en open(archivo1,...)",errno);
    perror("\nError en open");
    exit(-1);
}

umask(0);
if( (fd2=open("archivo2",O_CREAT|O_TRUNC|O_WRONLY,S_IRGRP|S_IWGRP|S_IXGRP))<0) {
    printf("\nError %d en open(archivo2,...)",errno);
    perror("\nError en open");
    exit(-1);
}

//CAMBIO DE PERMISOS
if(stat("archivo1",&atributos) < 0) {
    printf("\nError al intentar acceder a los atributos de archivo1");
    perror("\nError en lstat");
    exit(-1);
}
if(chmod("archivo1", (atributos.st_mode & ~S_IXGRP) | S_ISGID) < 0) {
    perror("\nError en chmod para archivo1");
    exit(-1);
}
if(chmod("archivo2",S_IRWXU | S_IRGRP | S_IWGRP | S_IROTH) < 0) {
    perror("\nError en chmod para archivo2");
    exit(-1);
}

return 0;
}
\end{cppcode}

{\color{blue} Crea dos archivos, \verb!archivo1! con permisos \verb!O_WRONLY,S_IRGRP|S_IWGRP|S_IXGRP!, es decir, \verb!---rwx---!, Este valor se logra con la siguiente operación sobre el \verb!umask!, que en mi máquina por defecto vale \verb!0002!:
\begin{cppcode}
(000000000100000 | 000000000010000 | 000000000001000) 
&           ~000000000000010 
= (000000000111000) & 111111111111101 
= 000000000111000
\end{cppcode}
 y luego se cambia el \verb!umask! a \verb!0! y se crea \verb!archivo2! con permisos \verb!S_IRGRP|S_IWGRP|S_IXGRP!, aplicando la máscara 0 se obtiene 
\begin{cppcode}
(000000000100000 | 000000000010000 | 000000000001000) 
&           ~000000000000000  
= 000000000111000 & 111111111111111 
= 000000000111000
\end{cppcode}
que es \verb!---rwx---!.

La segunda parte del programa, obtiene los atributos de \verb!archivo1!, y luego le cambia los permisos con \verb!(atributos.st_mode & ~S_IXGRP)!, que quiere decir: Si ya tiene permisos de ejecución para el grupo, se desactivan, en caso de no tenerlo, lo activa. Y \verb!| S_ISGID! activa el grupo efectivo.
}


\begin{exercise}
Realiza un programa en C utilizando las llamadas al sistema necesarias que
acepte como entrada:
\begin{itemize}
	\item Un argumento que representa el \emph{\textbf{'pathname'}} de un directorio.
    \item Otro argumento que es u\emph{n \textbf{número octal de 4 dígitos}} (similar al que se puede utilizar
para cambiar los permisos en la llamada al sistema chmod). Para convertir este
argumento tipo cadena a un tipo numérico puedes utilizar la función strtol. Consulta
el manual en línea para conocer sus argumentos.
\end{itemize}
El programa tiene que usar el número octal indicado en el segundo argumento para cambiar
los permisos de todos los archivos que se encuentren en el directorio indicado en el primer
argumento.
El programa debe proporcionar en la salida estándar una línea para cada archivo del
directorio que esté formada por:
\begin{bashcode}
<nombre_de_archivo> : <permisos_antiguos> <permisos_nuevos>
\end{bashcode}
Si no se pueden cambiar los permisos de un determinado archivo se debe especificar la
siguiente información en la línea de salida:
\begin{bashcode}
<nombre_de_archivo> : <errno> <permisos_antiguos>
\end{bashcode}

\cppscript{../Sesion2/src/ej2}{Ejercicio2}

\end{exercise}
%\chapter{Sesión 3}

\begin{exercise}
Implementa un programa en C que tenga como argumento un número entero.
Este programa debe crear un proceso hijo que se encargará de comprobar si dicho número es
un número par o impar e informará al usuario con un mensaje que se enviará por la salida
estándar. A su vez, el proceso padre comprobará si dicho número es divisible por 4, e
informará si lo es o no usando igualmente la salida estándar.
\cppscript{../Sesion3/src/ej2}{Ej2}
\end{exercise}
\begin{exercise}
¿Qué hace el siguiente programa? Intenta entender lo que ocurre con las
variables y sobre todo con los mensajes por pantalla cuando el núcleo tiene
activado/desactivado el mecanismo de buffering.
\end{exercise}

\begin{exercise}
Implementa un programa que lance cinco procesos hijo. Cada uno de ellos se
identificará en la salida estándar, mostrando un mensaje del tipo Soy el hijo PID. El
proceso padre simplemente tendrá que esperar la finalización de todos sus hijos y cada vez
que detecte la finalización de uno de sus hijos escribirá en la salida estándar un mensaje del
tipo:

\emph{Acaba de finalizar mi hijo con <PID>}\\
\emph{Sólo me quedan <NUM_HIJOS> hijos vivos}
\cppscript{../Sesion3/src/ej3}{Ej3}
\end{exercise}

\begin{exercise}
Implementa una modificación sobre el anterior programa en la que el proceso
padre espera primero a los hijos creados en orden impar (1o,3o,5o) y después a los hijos pares
(2o y 4o).
\cppscript{../Sesion3/src/ej4}{Ej4}
\end{exercise}

\begin{exercise}
¿Qué hace el siguiente programa?
\begin{cppcode}
//tarea5.c
//Trabajo con llamadas al sistema del Subsistema de Procesos conforme a POSIX 2.10

#include<sys/types.h>	
#include<sys/wait.h>	
#include<unistd.h>
#include<stdio.h>
#include<errno.h>
#include <stdlib.h>


int main(int argc, char *argv[])
{
pid_t pid;
int estado;

if( (pid=fork())<0) {
	perror("\nError en el fork");
	exit(-1);
}
else if(pid==0) {  //proceso hijo ejecutando el programa
	if( (execl("/usr/bin/ldd","ldd","./tarea5")<0)) {
		perror("\nError en el execl");
		exit(-1);
	}
}
wait(&estado);
printf("\nMi hijo %d ha finalizado con el estado %d\n",pid,estado);

exit(0);

}
\end{cppcode}
\emph{Crea un proceso hijo y éste último ejecuta el comando ldd sobre el archivo tarea8.}
\end{exercise}
\begin{exercise}
Escribe un programa que acepte como argumentos el nombre de un programa,
sus argumentos si los tiene, y opcionalmente la cadena “bg”. Nuesto programa deberá
ejecutar el programa pasado como primer argumento en foreground si no se especifica la
cadena “bg” y en background en caso contrario. Si el programa tiene argumentos hay que
ejecutarlo con éstos.
\cppscript{../Sesion3/src/ej6}{Ej6}
\end{exercise}

%\chapter{Sesión 4}

\begin{exercise}
Consulte en el manual las llamadas al sistema para la creación de archivos
especiales en general (mknod) y la específica para archivos FIFO (mkfifo). Pruebe a ejecutar
el siguiente código correspondiente a dos programas que modelan el problema del
productor/consumidor, los cuales utilizan como mecanismo de comunicación un cauce FIFO.
Determine en qué orden y manera se han de ejecutar los dos programas para su correcto
funcionamiento y cómo queda reflejado en el sistema que estamos utilizando un cauce FIFO.
Justifique la respuesta.
\cppscript{../Sesion4/consumidorFIFO}{consumidorFIFO}
\cppscript{../Sesion4/productorFIFO}{productorFIFO}

\emph{El orden da igual, es posible ejecutar primero el productor, quedando así éste
a la espera de que se ejecute el consumidor y realice una lectura sobre el pipe. O puede
ejecutarse primero en consumidor, quedando así a la espera de que se produzcan datos que
consumir.}

\emph{En el sistema queda reflejado el uso del cauce con la existencia del archivo}
\begin{bashcode}
prw-rw-rw-  1 hkr hkr      0 Dec 16 14:04 ComunicacionFIFO|
\end{bashcode}
\end{exercise}

\begin{exercise}
Consulte en el manual en línea la llamada al sistema pipe para la creación de
cauces sin nombre. Pruebe a ejecutar el siguiente programa que utiliza un cauce sin nombre y
describa la función que realiza. Justifique la respuesta.
\begin{cppcode}
/*
tarea6.c
Trabajo con llamadas al sistema del Subsistema de Procesos y Cauces conforme a
POSIX 2.10
*/
#include<sys/types.h>
#include<fcntl.h>
#include<unistd.h>
#include<stdio.h>
#include<stdlib.h>
#include<errno.h>
int main(int argc, char *argv[])
{
int fd[2], numBytes;
pid_t PID;
char mensaje[]= "\nEl primer mensaje transmitido por un cauce!!\n";
char buffer[80];
pipe(fd); // Llamada al sistema para crear un cauce sin nombre
if ( (PID= fork())<0) {
perror("fork");
exit(1);
}
if (PID == 0) {
//Cierre del descriptor de lectura en el proceso hijo
close(fd[0]);
// Enviar el mensaje a través del cauce usando el descriptor de escritura
write(fd[1],mensaje,strlen(mensaje)+1);
exit(0);
}
else { // Estoy en el proceso padre porque PID != 0
//Cerrar el descriptor de escritura en el proceso padre
close(fd[1]);
//Leer datos desde el cauce.
numBytes= read(fd[0],buffer,sizeof(buffer));
printf("\nEl número de bytes recibidos es: %d",numBytes);
printf("\nLa cadena enviada a través del cauce es: %s", buffer);
}
return(0);
}
\end{cppcode}
\emph{Se crea un cauce sin nombre y seguidamente un proceso hijo. En el proceso hijo se cierra el descriptor de lectura almacenado en \textbf{fd[0]}, ya que 
es el hijo que el que va a escribir en el cauce y por tanto debe de cerrar ese lado del cauce. En el proceso padre se realiza lo contrario, se cierra el descriptor de escritura del cauce, ya que éste sólo va a encargarse de leer lo que escriba el hijo.}
\cppscript{../Sesion4/tarea6}{Tarea6}
\end{exercise}

\begin{exercise}
Redirigiendo las entradas y salidas estándares de los procesos a los cauces
podemos escribir un programa en lenguaje C que permita comunicar órdenes existentes sin
necesidad de reprogramarlas, tal como hace el shell (por ejemplo ls | sort). En particular,
ejecute el siguiente programa que ilustra la comunicación entre proceso padre e hijo a través
de un cauce sin nombre redirigiendo la entrada estándar y la salida estándar del padre y el
hijo respectivamente.
\cppscript{../Sesion4/tarea7}{Tarea7}
\end{exercise}

\begin{exercise}
Compare el siguiente programa con el anterior y ejecútelo. Describa la
principal diferencia, si existe, tanto en su código como en el resultado de la ejecución.
\cppscript{../Sesion4/tarea8}{Tarea8}
\emph{La única diferencia es que se usa la llamada al sistema \textbf{dup2}, que realiza simultáneamente las llamadas a \textbf{close} y \textbf{dup}.}
\end{exercise}

\begin{exercise}
Este ejercicio se basa en la idea de utilizar varios procesos para realizar partes de una
computación en paralelo. Para ello, deberá construir un programa que siga el esquema de
computación maestro-esclavo, en el cual existen varios procesos trabajadores (esclavos)
idénticos y un único proceso que reparte trabajo y reúne resultados (maestro). Cada esclavo
es capaz de realizar una computación que le asigne el maestro y enviar a este último los
resultados para que sean mostrados en pantalla por el maestro.

El ejercicio concreto a programar consistirá en el cálculo de los números primos que hay en
un intervalo. Será necesario construir dos programas, maestro y esclavo, que darán lugar a
tres procesos, 1 maestro y 2 esclavos. Además se tendrá que tener en cuenta la siguiente
especificación:

\begin{itemize}
    \item El intervalo de números naturales donde calcular los número primos se pasará como
argumento al programa maestro. El maestro creará dos procesos esclavos y dividirá
el intervalo en dos subintervalos de igual tamaño pasando cada subintervalo como
argumento a cada programa esclavo. Por ejemplo, si al maestro le proporcionamos el
intervalo entre 1000 y 2000, entonces un esclavo debe calcular y devolver los
números primos comprendidos en el subintervalo entre 1000 y 1500, y el otro esclavo
entre 1501 y 2000. El maestro irá recibiendo y mostrando en pantalla (también uno a
uno) los números primos calculados por los esclavos en orden creciente.
    \item El programa esclavo tiene como argumentos el extremo inferior y superior del
intervalo sobre el que buscará números primos. Para identificar un número primo
utiliza el siguiente método concreto: un número n es primo si no es divisible por
ningún k tal que 2 < k < sqrt(n), donde sqrt corresponde a la función de cálculo de la
raíz cuadrada (consulte dicha función en el manual). El esclavo envía al maestro cada
primo encontrado como un dato entero (4 bytes) que escribe en la salida estándar, la
cuál se tiene que encontrar redireccionada a un cauce sin nombre. Los dos cauces sin
nombre necesarios, cada uno para comunicar cada esclavo con el maestro, los creará
el maestro inicialmente. Una vez que un esclavo haya calculado y enviado (uno a uno)
al maestro todos los primos en su correspondiente intervalo terminará.
\end{itemize}
\cppscript{../Sesion4/definitivo}{Ejercicio5}
\end{exercise}

%\section{Sesión 5}

\begin{exercise}
Compila y ejecuta los siguientes programas y trata de entender su funcionamiento.
\end{exercise}

\begin{exercise}
Escribe un programa en C llamado contador, tal que cada vez que reciba una señal
que se pueda manejar, muestre por pantalla la señal y el número de veces que se ha recibido ese
tipo de señal, y un mensaje inicial indicando las señales que no puede manejar. En el cuadro
siguiente se muestra un ejemplo de ejecución del programa
\cppscript{../Sesion5/src/contador}{Contador}
\end{exercise}

\section{Sesión 6}

\begin{exercise}
Implementa un programa que admita tres argumentos. El primer argumento será
una orden de Linux; el segundo, uno de los siguientes caracteres “<” o “>”, y el tercero el
nombre de un archivo (que puede existir o no). El programa ejecutará la orden que se
especifica como argumento primero e implementará la redirección especificada por el
segundo argumento hacia el archivo indicado en el tercer argumento. Por ejemplo, si
deseamos redireccionar la salida estándar de sort a un archivo temporal, ejecutaríamos (el
carácter de redirección > lo ponemos entrecomillado para que no lo interprete el shell y se
coja como argumento del programa):
\cppscript{../Sesion6/src/ej1}{Ejercicio 1}
\end{exercise}

%%%%%%%%%%%%%%%%%%%%%%%%%%%%%%%%%%%%%%%%%%%%%%%%
%%%%%%%%%%%%%%%%% T1 Estructuras SO %%%%%%%%%%%%
%%%%%%%%%%%%%%%%%%%%%%%%%%%%%%%%%%%%%%%%%%%%%%%%


\end{document}
