\chapter{Sesión 1}

\begin{exercise}
?`Qué hace el siguiente programa? Probad tras la ejecución del programa las
siguientes órdenes del shell: \verb!$>cat archivo y $> od -c archivo!
\begin{cppcode}
/*
tarea1.c
Trabajo con llamadas al sistema del Sistema de Archivos ''POSIX 2.10 compliant''
Probar tras la ejecución del programa: >cat archivo y > od -c archivo
*/

#include<sys/types.h>   //Primitive system data types for abstraction of implementation-dependent data types.
                        //POSIX Standard: 2.6 Primitive System Data Types <sys/types.h>
#include<sys/stat.h>
#include<fcntl.h>
#include<stdlib.h>
#include<stdio.h>
#include<errno.h>

char buf1[]="abcdefghij";
char buf2[]="ABCDEFGHIJ";

int main(int argc, char *argv[])
{
int fd;

if( (fd=open("archivo",O_CREAT|O_TRUNC|O_WRONLY,S_IRUSR|S_IWUSR))<0) {
    printf("\nError %d en open",errno);
    perror("\nError en open");
    exit(-1);
}
if(write(fd,buf1,10) != 10) {
    perror("\nError en primer write");
    exit(-1);
}

if(lseek(fd,40,SEEK_SET) < 0) {
    perror("\nError en lseek");
    exit(-1);
}

if(write(fd,buf2,10) != 10) {
    perror("\nError en segundo write");
    exit(-1);
}

return 0;
}
\end{cppcode}

{\color{blue} El programa crea un archivo llamado \verb!archivo!, escribe el primer buffer, que contiene \verb!abcdefghij!, éstos caracteres se escriben desde la primera posición del archivo, luego, mueve el puntero a la posición 40 y escribe el segundo buffer, que contiene \verb!ABCDEFGHIJ!, por tanto, el fichero va a tener contenido nulo en las posiciones 11-39. Con el comando \verb!cat! no se aprecia el contenido nulo porque es ignodaro, sin embargo con od sí se muestran:}

\begin{bashcode}
 od -c archivo 
0000000   a   b   c   d   e   f   g   h   i   j  \0  \0  \0  \0  \0  \0
0000020  \0  \0  \0  \0  \0  \0  \0  \0  \0  \0  \0  \0  \0  \0  \0  \0
0000040  \0  \0  \0  \0  \0  \0  \0  \0   A   B   C   D   E   F   G   H
0000060   I   J
0000062
\end{bashcode}

\end{exercise}

\begin{exercise}
Implementa un programa que acepte como argumento un ''pathname'', abra el
archivo correspondiente y utilizando un tamaño de lectura en bloques de 80 Bytes cree un
archivo de salida en el que debe aparecer lo siguiente:
\begin{bashcode}
Bloque 1
//los primeros 80 Bytes
Bloque 2
//los siguientes 80 Bytes
...
Bloque n
//los siguientes 80 Bytes
\end{bashcode}
\cppscript{../Sesion1/src/ej2}{Ejercicio 2}

?`Cómo tendrías que modificar el programa para que una vez
finalizada la escritura en el archivo de salida y antes de cerrarlo, pudiésemos indicar en su
primera línea el número de etiquetas ''Bloque i'' escritas de forma que tuviese la siguiente
apariencia?:

{\color{blue} Implementado en la versión de arriba}
\end{exercise}

\begin{exercise}
?`Qué hace el siguiente programa?

\begin{cppcode}
/*
tarea2.c
Trabajo con llamadas al sistema del Sistema de Archivos ''POSIX 2.10 compliant''
*/

#include<sys/types.h>    //Primitive system data types for abstraction of implementation-dependent data types.
                        //POSIX Standard: 2.6 Primitive System Data Types <sys/types.h>
#include<unistd.h>        //POSIX Standard: 2.10 Symbolic Constants         <unistd.h>
#include<sys/stat.h>
#include<stdio.h>
#include<errno.h>
#include<string.h>

int main(int argc, char *argv[])
{
int i;
struct stat atributos;
char tipoArchivo[30];

if(argc<2) {
    printf("\nSintaxis de ejecucion: tarea2 [<nombre_archivo>]+\n\n");
    exit(-1);
}
for(i=1;i<argc;i++) {
    printf("%s: ", argv[i]);
    if(lstat(argv[i],&atributos) < 0) {
        printf("\nError al intentar acceder a los atributos de %s",argv[i]);
        perror("\nError en lstat");
    }
    else {
        if(S_ISREG(atributos.st_mode)) strcpy(tipoArchivo,"Regular");
        else if(S_ISDIR(atributos.st_mode)) strcpy(tipoArchivo,"Directorio");
        else if(S_ISCHR(atributos.st_mode)) strcpy(tipoArchivo,"Especial de caracteres");
        else if(S_ISBLK(atributos.st_mode)) strcpy(tipoArchivo,"Especial de bloques");
        else if(S_ISFIFO(atributos.st_mode)) strcpy(tipoArchivo,"Tuber�a con nombre (FIFO)");
        else if(S_ISLNK(atributos.st_mode)) strcpy(tipoArchivo,"Enlace relativo (soft)");
        else if(S_ISSOCK(atributos.st_mode)) strcpy(tipoArchivo,"Socket");
        else strcpy(tipoArchivo,"Tipo de archivo desconocido");
        printf("%s\n",tipoArchivo);
    }
}

return 0;
}
\end{cppcode}

{\color{blue} 
Para cada archivo que se le pase como parámetro, muestra qué tipo de ficheros son, en este caso usando la llamada \verb/lstat/, que se diferencia de \verb/stat/ en que si el archivo es un enlace simbólico, \verb/lstat/ proporciona los atributos del enlace en sí, no del archivo al que referencia, ejemplo de ejecución:

\begin{bashcode}
./tarea2 *
dump: Regular
ej2: Regular
ej2.c: Regular
ej2.c~: Regular
fichero: Regular
fichero~: Regular
tarea1.c: Regular
tarea2: Regular
tarea2.c: Regular
tarea3.c: Regular
\end{bashcode}
}
\end{exercise}

\begin{exercise}
Define una macro en lenguaje C que tenga la misma funcionalidad que la macro
S_ISREG(mode) usando para ello los flags definidos en <sys/stat.h> para el campo st_mode
de la struct stat, y comprueba que funciona en un programa simple. Consulta en un libro
de C o en internet cómo se especifica una macro con argumento en C.

\cppscript{../Sesion1/src/ej4}{Ejercicio 4}
\end{exercise}