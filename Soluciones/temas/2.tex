\chapter{Sesión 2}

\begin{exercise}
?`Qué hace el siguiente programa?

\end{exercise}
\begin{cppcode}
/*
tarea3.c
Trabajo con llamadas al sistema del Sistema de Archivos ''POSIX 2.10 compliant''
Este programa fuente está pensado para que se cree primero un programa con la parte
 de CREACION DE ARCHIVOS y se haga un ls -l para fijarnos en los permisos y entender
 la llamada umask.
En segundo lugar (una vez creados los archivos) hay que crear un segundo programa
 con la parte de CAMBIO DE PERMISOS para comprender el cambio de permisos relativo
 a los permisos que actualmente tiene un archivo frente a un establecimiento de permisos
 absoluto.
*/

#include<sys/types.h>   //Primitive system data types for abstraction of implementation-dependent data types.
                        //POSIX Standard: 2.6 Primitive System Data Types <sys/types.h>
#include<unistd.h>      //POSIX Standard: 2.10 Symbolic Constants         <unistd.h>
#include<sys/stat.h>
#include<fcntl.h>       //Needed for open
#include<stdio.h>
#include<errno.h>


int main(int argc, char *argv[])
{
int fd1,fd2;
struct stat atributos;

//CREACION DE ARCHIVOS
if( (fd1=open("archivo1",O_CREAT|O_TRUNC|O_WRONLY,S_IRGRP|S_IWGRP|S_IXGRP))<0) {
    printf("\nError %d en open(archivo1,...)",errno);
    perror("\nError en open");
    exit(-1);
}

umask(0);
if( (fd2=open("archivo2",O_CREAT|O_TRUNC|O_WRONLY,S_IRGRP|S_IWGRP|S_IXGRP))<0) {
    printf("\nError %d en open(archivo2,...)",errno);
    perror("\nError en open");
    exit(-1);
}

//CAMBIO DE PERMISOS
if(stat("archivo1",&atributos) < 0) {
    printf("\nError al intentar acceder a los atributos de archivo1");
    perror("\nError en lstat");
    exit(-1);
}
if(chmod("archivo1", (atributos.st_mode & ~S_IXGRP) | S_ISGID) < 0) {
    perror("\nError en chmod para archivo1");
    exit(-1);
}
if(chmod("archivo2",S_IRWXU | S_IRGRP | S_IWGRP | S_IROTH) < 0) {
    perror("\nError en chmod para archivo2");
    exit(-1);
}

return 0;
}
\end{cppcode}

{\color{blue} Crea dos archivos, \verb!archivo1! con permisos \verb!O_WRONLY,S_IRGRP|S_IWGRP|S_IXGRP!, es decir, \verb!---rwx---!, Este valor se logra con la siguiente operación sobre el \verb!umask!, que en mi máquina por defecto vale \verb!0002!:
\begin{cppcode}
(000000000100000 | 000000000010000 | 000000000001000) 
&           ~000000000000010 
= (000000000111000) & 111111111111101 
= 000000000111000
\end{cppcode}
 y luego se cambia el \verb!umask! a \verb!0! y se crea \verb!archivo2! con permisos \verb!S_IRGRP|S_IWGRP|S_IXGRP!, aplicando la máscara 0 se obtiene 
\begin{cppcode}
(000000000100000 | 000000000010000 | 000000000001000) 
&           ~000000000000000  
= 000000000111000 & 111111111111111 
= 000000000111000
\end{cppcode}
que es \verb!---rwx---!.

La segunda parte del programa, obtiene los atributos de \verb!archivo1!, y luego le cambia los permisos con \verb!(atributos.st_mode & ~S_IXGRP)!, que quiere decir: Si ya tiene permisos de ejecución para el grupo, se desactivan, en caso de no tenerlo, lo activa. Y \verb!| S_ISGID! activa el grupo efectivo.
}


\begin{exercise}
Realiza un programa en C utilizando las llamadas al sistema necesarias que
acepte como entrada:
\begin{itemize}
	\item Un argumento que representa el \emph{\textbf{'pathname'}} de un directorio.
    \item Otro argumento que es u\emph{n \textbf{número octal de 4 dígitos}} (similar al que se puede utilizar
para cambiar los permisos en la llamada al sistema chmod). Para convertir este
argumento tipo cadena a un tipo numérico puedes utilizar la función strtol. Consulta
el manual en línea para conocer sus argumentos.
\end{itemize}
El programa tiene que usar el número octal indicado en el segundo argumento para cambiar
los permisos de todos los archivos que se encuentren en el directorio indicado en el primer
argumento.
El programa debe proporcionar en la salida estándar una línea para cada archivo del
directorio que esté formada por:
\begin{bashcode}
<nombre_de_archivo> : <permisos_antiguos> <permisos_nuevos>
\end{bashcode}
Si no se pueden cambiar los permisos de un determinado archivo se debe especificar la
siguiente información en la línea de salida:
\begin{bashcode}
<nombre_de_archivo> : <errno> <permisos_antiguos>
\end{bashcode}

\cppscript{../Sesion2/src/ej2}{Ejercicio2}

\end{exercise}