\chapter{Sesión 3}

\begin{exercise}
Implementa un programa en C que tenga como argumento un número entero.
Este programa debe crear un proceso hijo que se encargará de comprobar si dicho número es
un número par o impar e informará al usuario con un mensaje que se enviará por la salida
estándar. A su vez, el proceso padre comprobará si dicho número es divisible por 4, e
informará si lo es o no usando igualmente la salida estándar.
\cppscript{../Sesion3/src/ej2}{Ej2}
\end{exercise}
\begin{exercise}
¿Qué hace el siguiente programa? Intenta entender lo que ocurre con las
variables y sobre todo con los mensajes por pantalla cuando el núcleo tiene
activado/desactivado el mecanismo de buffering.
\end{exercise}

\begin{exercise}
Implementa un programa que lance cinco procesos hijo. Cada uno de ellos se
identificará en la salida estándar, mostrando un mensaje del tipo Soy el hijo PID. El
proceso padre simplemente tendrá que esperar la finalización de todos sus hijos y cada vez
que detecte la finalización de uno de sus hijos escribirá en la salida estándar un mensaje del
tipo:

\emph{Acaba de finalizar mi hijo con <PID>}\\
\emph{Sólo me quedan <NUM_HIJOS> hijos vivos}
\cppscript{../Sesion3/src/ej3}{Ej3}
\end{exercise}

\begin{exercise}
Implementa una modificación sobre el anterior programa en la que el proceso
padre espera primero a los hijos creados en orden impar (1o,3o,5o) y después a los hijos pares
(2o y 4o).
\cppscript{../Sesion3/src/ej4}{Ej4}
\end{exercise}

\begin{exercise}
¿Qué hace el siguiente programa?
\begin{cppcode}
//tarea5.c
//Trabajo con llamadas al sistema del Subsistema de Procesos conforme a POSIX 2.10

#include<sys/types.h>	
#include<sys/wait.h>	
#include<unistd.h>
#include<stdio.h>
#include<errno.h>
#include <stdlib.h>


int main(int argc, char *argv[])
{
pid_t pid;
int estado;

if( (pid=fork())<0) {
	perror("\nError en el fork");
	exit(-1);
}
else if(pid==0) {  //proceso hijo ejecutando el programa
	if( (execl("/usr/bin/ldd","ldd","./tarea5")<0)) {
		perror("\nError en el execl");
		exit(-1);
	}
}
wait(&estado);
printf("\nMi hijo %d ha finalizado con el estado %d\n",pid,estado);

exit(0);

}
\end{cppcode}
\emph{Crea un proceso hijo y éste último ejecuta el comando ldd sobre el archivo tarea8.}
\end{exercise}
\begin{exercise}
Escribe un programa que acepte como argumentos el nombre de un programa,
sus argumentos si los tiene, y opcionalmente la cadena “bg”. Nuesto programa deberá
ejecutar el programa pasado como primer argumento en foreground si no se especifica la
cadena “bg” y en background en caso contrario. Si el programa tiene argumentos hay que
ejecutarlo con éstos.
\cppscript{../Sesion3/src/ej6}{Ej6}
\end{exercise}
